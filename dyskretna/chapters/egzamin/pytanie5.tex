\section{Funkcje tworzące}

\subsection{Definicja}

\begin{definition}
    \textbf{Funkcją tworzącą} ciągu \((a_n)_{n \in \natural}\) nazywamy funkcję postaci:
    \begin{equation*}
        A(x) = \sum_{n=0}^{\infty} a_n x^n
    \end{equation*}
\end{definition}

\subsection{Rozwinięcie w szereg \(\frac{1}{(1-ax^k)^m}\)}

Bazowa funkcja tworząca ciągu \((1)_{n \in \natural}\) to \(\frac{1}{1-x}\). Dowód jest trywialny:

\begin{proof}
    Niech ciag \((a_n)_{n \in \natural}\) będzie ciągiem jedynkowym. Wtedy funkcja tworząca tego ciągu to:

    $$ A(x) = \sum_{n=0}^{\infty} x^n $$
    $$ A(x) x = \sum_{n=0}^{\infty} x^{n+1} $$
    $$ A(x) x = \sum_{n=1}^{\infty} x^n $$
    $$ A(x) - A(x) x = 1 $$
    $$ A(x) (1 - x) = 1 $$
    $$ A(x) = \dfrac{1}{1-x} $$
\end{proof}

Niech $G(x) = \dfrac{1}{(1-ax^k)^m}$ oraz $F(x) = \dfrac{1}{(1-ax)^m}$. Wtedy prosto wynika z powyższego, że $G(x) = F(x^k)$, więc możemy rozważac tylko $F(x)$.

\begin{lemma}
    Funkcja tworząca ciągu \((\binom{n+m-1}{m-1} a^n )_{n \in \natural}\) to \(\frac{1}{(1-ax)^m}\) dla \( m \in \natural_1, a \in \mathbb{C} \setminus \{0\} \).
\end{lemma}

\begin{proof}
    Udowodnimy twierdzenie indukcyjnie. Dla $m = 1$ mamy:

    $$ \sum_{n=0}^{\infty} \binom{n+1-1}{1-1} a^n x^n = \sum_{n=0}^{\infty} a^n x^n = \dfrac{1}{1-ax} $$

    Załóżmy, że twierdzenie zachodzi dla $m$. Wtedy udowodnimy dla $m+1$:

    \begin{multiline}
        $$ \frac{1}{(1-ax)^{m+1}} = \frac{1}{(1-ax)^m} \cdot \frac{1}{(1-ax)} = \left( \sum_{n=0}^{\infty} \binom{n+m-1}{m-1} a^n x^n \right) \cdot \frac{1}{1-ax} = $$
        $$ \sum_{n=0}^{\infty} \binom{n+m-1}{m-1} a^n x^n \cdot \sum_{n=0}^{\infty} a^n x^n = \sum_{n=0}^{\infty} \left(\sum_{k=0}^{n} \binom{k+m-1}{m-1} a^k a^{n-k}\right) x^n = $$
        $$ = \sum_{n=0}^{\infty} a^{n} x^n \left( \sum_{k=0}^{n} \binom{k+m-1}{m-1} \right) = \sum_{n=0}^{\infty} a^{n} x^n \binom{n+m}{m} $$
    \end{multiline}
\end{proof}

Jeszce nie jest oczywistym przedostatni krok w powyższym dowodzie.

\begin{lemma}
    \( \displaystyle \sum_{k=0}^{n} \binom{k+m-1}{m-1} = \binom{n+m}{m} \) dla \( n, m \in \natural \).
\end{lemma}

\begin{proof}
    Udowodnimy to twierdzenie indukcyjnie. Dla \( n = 0 \) mamy:

    $$ \sum_{k=0}^{0} \binom{k+m-1}{m-1} = \binom{m-1}{m-1} = 1 = \binom{m}{m} $$

    Załóżmy, że twierdzenie zachodzi dla \( n \). Wtedy udowodnimy dla \( n+1 \):

    \begin{multiline}
        $$ \sum_{k=0}^{n+1} \binom{k+m-1}{m-1} = \sum_{k=0}^{n} \binom{k+m-1}{m-1} + \binom{n+m}{m-1} = \binom{n+m}{m} + \binom{n+m}{m-1} = \binom{n+m+1}{m} $$
    \end{multiline}
\end{proof}


\subsection{Operacje na funkcjach tworzących}

Niech \(A(x)\) będzie funkcją tworzącą ciągu \((a_n) n \in \natural\).
Wyprowadzić funkcję tworzące ciągów: \((0, \ldots , 0, a_0, a_1, \ldots)\), \((a_k, a_k+1, \ldots)\), \((\sum_{n=0}^k a_n)k \in \natural\).

Żeby wyprowadzić funkcję tworzącą ciągu \((0, \ldots , 0, a_0, a_1, \ldots)\) trzeba przemnożyć \(A(x)\) przez \(x^k\), gdzie \(k\) to liczba zer na początku ciągu.

Żeby wyprowadzić funkcję tworzącą ciągu \((a_k, a_k+1, \ldots)\) trzeba zrobić \( B(x) = \frac{A(x) - a_0 - a_1 x - \ldots - a_{k-1} x^{k-1}}{x^k} \).

Żeby wyprowadzić funkcję tworzącą ciągu \((\sum_{n=0}^k a_n)k \in \natural\) trzeba przemnożyć \(A(x)\) przez \(\frac{1}{1-x}\).

\begin{proof}
    Niech \(A(x)\) będzie funkcją tworzącą ciągu \((a_n) n \in \natural\). Wtedy funkcja tworząca ciągu \(S(x) = \sum_{n=0}^k a_n \) to $S(x) = \frac{A(x)}{1-x}$:

    $$ S(x) = \frac{A(x)}{1-x} = A(x) \cdot \frac{1}{1-x} = A(x) \cdot \left(\sum_{n=0}^{\infty} x^n\right) = \sum_{n=0}^{\infty} \left(\sum_{k=0}^{n} a_k \cdot 1 \right) x^n = $$
    $$ = \sum_{n=0}^{\infty} \left(\sum_{k=0}^{n} a_k \right) x^n $$

\end{proof}
   
\subsection{Uogólnione współczynniki dwumianowe}

\begin{definition}
    \textbf{Uogólnionym współczynnikiem dwumianowym} nazywamy liczbę postaci:
    \begin{equation*}
        \binom{a}{n} = \frac{a(a-1)(a-2) \ldots (a-n+1)}{n!}
    \end{equation*}
\end{definition}

\subsection{Funkcja tworząca ciągu Fibonacciego}

% add refference to already existing section "Ciąg Fibbonaciego"

Już było w sekcji \ref{sec:gen:fibonacci}.

\subsection{Wzór zwarty na liczby Fibonacciego}

Już było w sekcji \ref{sec:gen:fibonacci} (Może być to będzie rospisane lepiej pózniej).