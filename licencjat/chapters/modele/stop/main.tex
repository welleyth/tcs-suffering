\subsection{Uniwersalna maszyna Turinga}

By mówić o problemie stopu, musimy najpierw wprowadzić definicję uniwersalnej Maszyny Turinga -- to znaczy takiej, że jest w stanie ,,zasymulować'' działanie dowolnej innej MT za pomocą jakiegoś kodowania. 

Niech \( M = (Q, \Sigma, \Gamma, \delta, q_0, \blank, F) \) będzie (deterministyczną) maszyną którą chcemy symulować. Pokażemy jak zakodować \( M \) nad alfabetem \( \Sigma_U = \set{0, 1} \).

\begin{itemize}
    \item Niech \( Q = \set{q_1, \dots, q_n} \)
    
    Kodujemy \( q_i \) jako ciąg \( i \) jedynek.
    
    \item \( \blank \) będziemy kodować jako pojedynczą jedynkę.
    
    
    \item Podobnie niech \( \Sigma = \set{a_1, \dots, a_n} \)
    
    Kodujemy \( a_i \) jako ciąg \( i + 1 \) jedynek.
    
    \item Dla \( \Gamma = \set{z_1, \dots z_n} \)
    
    Kodujemy \( z_i \) jako ciąg \( i + \card{\Sigma + 1} \) jedynek -- dla odróżnienia od \( \Sigma \).
    
    \item \( \delta \) jest funkcją, czyli zbiorem par \( ((q_1, z_1), (q_2, z_2, \pm 1) \)
    
    Parę \( ((q_i, z_j), (q_k, z_l), 1) \) zakodujemy jako
    
    \[
        \underbrace{1 \dots 1}_i
        0
        \underbrace{1 \dots 1}_j
        0
        \underbrace{1 \dots 1}_k
        0
        \underbrace{1 \dots 1}_l
        0
        1
    \]
    
    Natomiast parę \( ((q_i, z_j), (q_k, z_l), -1) \) zakodujemy jako
    
    \[
        \underbrace{1 \dots 1}_i
        0
        \underbrace{1 \dots 1}_j
        0
        \underbrace{1 \dots 1}_k
        0
        \underbrace{1 \dots 1}_l
        0
        11
    \]
    
    Całą deltę kodujemy kodując wszystkie jej pary (w dowolnej kolejności) i oddzielając je dwoma zerami.
    
    \item \( F = \set{q_{i_1}, \dots q_{i_n}} \) kodujemy jako
    \[
        \underbrace{1 \dots 1}_{i_1} 0 \dots 0 \underbrace{1 \dots 1}_{i_n}
    \]
\end{itemize}

Całą maszynę \( M \) zapisujemy wpisując \( \delta, q_0, F \) oddzielając je ciągiem \( 000 \).

Słowo \( w = a_{i_1} \dots a_{i_n} \) kodujemy jako
\[
    \underbrace{1 \dots 1}_{i_1} 0 \dots 0 \underbrace{1 \dots 1}_{i_n}
\]

Teraz musimy jeszcze powiedzieć jak kodujemy konfigurację maszyny \( M \) a robimy to dość prosto --
Konfigurację \( X_1 \dots X_k q_i X_{k+1} \dots X_n \) kodujemy jako
\[
    0\underbrace{1 \dots 1}_{\text{kod } X_1} 0 \underbrace{1 \dots 1}_{\text{kod } X_k} 00 \underbrace{1 \dots 1}_{\text{kod } q_i} 00 \underbrace{1 \dots 1}_{\text{kod } X_{k + 1}} 0 \underbrace{1 \dots 1}_{\text{kod } X_n} 0
\]

Możemy założyć, że kodowanie \( M \) jest zapisane na taśmie wejściowej po której nigdy nie piszemy, zaś konfigurację \( M \) będziemy trzymać na osobnej taśmie.

Krok symulacji przebiega teraz następująco:
\begin{enumerate}
    \item Zlokalizuj głowicę \( M \) (szukamy \( 00 \))
    \item Znajdź w kodowaniu \( \delta \) wpis odpowiadający przejściu na \( q_i X_k \)
    \item Wykonaj zadane przejście -- możemy wpisać nową konfigurację na osobną, tymczasową taśmę i przepisać ją z powrotem na taśmę na której operujemy.
    
\end{enumerate}


\subsection{Problem stopu}
\label{lhp}
\begin{definition}
    Definiujemy język \( L_{HALT} \) następująco:
    \[
        L_{HALT} = \set{(M, w) : M \text{ zatrzymuje się na } w}
    \]
\end{definition}

Oczywiście, pisząc \(M\) mamy na myśli pewne kodowanie jakiejś Maszyny Turinga z którym jesteśmy w stanie pracować. 

\begin{theorem}
    \( L_{HALT} \in \re \setminus \r \)
\end{theorem}
\begin{proof}
    Oczywiście \( L_{HALT} \in \re \) bo możemy po prostu zasymulować \( M \) na \( w \) i jeśli się zatrzymamy to odpowiedź jest ,,TAK'' i wszystko super. Jeśli zaś odpowiedź to ,,NIE'' to nie musimy się w ogóle zatrzymywać.
    
    Aby pokazać że \( L_{HALT} \notin \r \) zakładamy nie wprost, że istnieje \( M \) z własnością stopu, taka że \( L(M) = L_{HALT} \).
    
    Konstruujemy teraz \( M' \) która:
    \begin{enumerate}
        \item wczytuje wejście \( x \)
        \item symuluje \( M \) na \( (x, x) \)
        \item jeśli \( M \) odpowiedziała ,,TAK'' to wpadamy w nieskończoną pętlę
        \item w przeciwnym razie zatrzymujemy się w dowolnym stanie
    \end{enumerate}
    
    Wrzucamy teraz do \( M \) wejście \( (M', M') \)
    
    Mamy dwie sytuacje:
    \begin{itemize}
        \item \( M \) zaakceptowała \( (M', M') \)
        
        Skoro \( (M', M') \in L(M) = L_{HALT} \) to oznacza, że
        wykonaliśmy krok (4) co ma miejsce jedynie gdy \( (x, x) \notin L(M) \). Tak się składa że u nas \( x = M' \) czyli \( (M', M') \notin L(M) \).
        
        \item \( M \) odrzuciła \( (M', M') \)
        
        Tutaj sytuacja jest podobna -- skoro \( (M', M') \notin L(M) = L_{HALT} \) to
        wykonaliśmy krok (3) co ma miejsce jedynie gdy \( (x, x) \in L(M) \), a ponieważ \( x = M' \) to \( (M', M') \in L(M) \).
    \end{itemize}
    
    Łącząc oba powyższe dostajemy \( (M', M') \in L(M) \iff (M', M') \notin L(M) \) co jest oczywiście sprzeczne. 
    
    W takim razie nie istnieje \( M \) z własnością stopu rozpoznająca \( L_{HALT} \), zatem \( L_{HALT} \notin \r \).
\end{proof}

\subsection{Dopełnienie problemu stopu} 
\begin{theorem}
\(\complement L_{HALT} \not \in \re\)
\end{theorem}

Powyższe twierdzenie wynika wprost z poniższego lematu:

\begin{lemma}
    Jeśli \( L \in RE \setminus R \), to \( \complement{L} \not \in RE\) 
\end{lemma}
\begin{proof}
    Załóżmy, że \( L \in RE \setminus R \) i \( \complement{L} \in RE\). 
    Mamy więc maszynę \( M \) rozpoznającą \( L \) oraz \( N \) rozpoznającą \( \complement{L} \)
    
    Konstruujemy DMT \( M' \) która:
    \begin{enumerate}
        \item Wczytuje wejście \( w \)
        \item Aż do akceptacji:
        \begin{enumerate}
            \item wykonaj jeden krok symulacji \( M \) na \( w \)
            \item wykonaj jeden krok symulacji \( N \) na \( w \)
        \end{enumerate}
        \item Jeśli \( M \) zaakceptowało to wypisz ,,TAK''
        \item Jeśli \( N \) zaakceptowało to wypisz ,,NIE''
    \end{enumerate}
    
    Oczywiście \( w \in L \lor w \in \complement{L} \) więc albo \( M \) zaakceptuje \( w \) albo zrobi to \( N \) -- któraś w końcu musi.
    Stanie się to po skończonej liczbie kroków niezależnie od \( w \) zatem \( L(M') = M \) oraz \( M' \) ma własność stopu.
    
    Trochę przypał bo w takim razie \( L \in (RE \setminus R) \cap R \) czyli nie istnieje.
    
\end{proof}