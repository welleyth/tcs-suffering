\begin{definition}
    Język \( L \) jest w klasie P (polynomial) jeśli istnieje wielomian \( p \)
    oraz Deterministyczna Maszyna Turinga działająca w czasie \( p \) taka, że \( L(M) = L \)
\end{definition}

\begin{theorem}
Język pusty jest w \(\p\). 
\end{theorem}
\begin{proof}
To było mało ambitne twierdzenie. Maszyna rozpoznająca ten język od razu przechodzi do stanu odrzucającego, a tym samym działa w czasie stałym.
\end{proof}

\begin{theorem}
\( \textsc{DNF-SAT} \in \p \).
\end{theorem}
\begin{proof}
To może przez chwilę szokować i powodować ból mózgu, ale, uwaga, jeśli problem \textsc{SAT} jest w postaci \textsc{DNF} to możemy to rozwiązać w czasie wielomianowym.

By to zrozumieć, wypada zobaczyć przykład formuły w \textsc{DNF}:

\[ 
            \overbrace{(x_1 \land \neg x_2 \land \neg x_3)}^{\text{Term DNF}} \lor \overbrace{(x_1 \land x_4)}^{\text{Term DNF}} \lor \overbrace{(\neg x_5)}^{\text{Term DNF}}    
\] 

Obserwujemy, że aby formuła w \textsc{DNF} była spełnialna, \textit{któryś} z jej termów musi być spełnialny. 

...a kiedy ciąg koniunkcji \textit{nie} jest spełnialny?

\textbf{Wtedy i tylko wtedy}, gdy zawiera w sobie jakąś zmienną i jej zaprzeczenie! Uzasadnienie tego faktu jest proste:

\begin{enumerate}
    \item Jeśli w ciągu koniunkcji występuje zmienna i jej zaprzeczenie to nie istnieje takie wartościowanie by ten ciąg koniunkcji był prawdziwy, siłą rzeczy;
    \item jeśli w ciągu koniunkcji \textit{nie} występuje zmienna wraz z jej zaprzeczeniem, to wszystkie zmienne które występują zanegowane wartościujemy na fałsz, a wszystkie które występują niezanegowane wartościujemy na prawdę. W ten sposób otrzymujemy spełnialny term. 
\end{enumerate}

Zauważamy, że ten warunek jest (z perspektywy algorytmicznej) trywialny do zbadania w czasie wielomianowym od rozmiaru wejścia. Badamy więc ten warunek dla każdej klauzuli po kolei -- jeśli którakolwiek z nich jest spełnialna, to cała formuła jest spełnialna. Ale super. 

\end{proof}

\begin{theorem}
2-SAT jest w \(\p\).
\end{theorem}
\begin{proof}
To jest bardziej ambitne twierdzenie. Algorytm znamy z ASDów, ale akurat w wymaganiach egzaminacyjnych do ASD go nie ma (a więc zostanie przedstawiony tutaj). 

Fundamentalna obserwacja w tym algorytmie opiera się na fakcie, że:

\[ 
p \lor q \equiv (\neg p \implies q) \land (\neg q \implies p)
\]

Aby popchnąć rozwiązanie tego problemu dalej, użyjemy \textit{teorii grafów}.

Tworzymy sobie graf \textit{literałów} dla formuły, którą dostaliśmy na wejściu. Na przykład dla takiej formuły:

\[
    (x_1 \lor \neg x_2) \land (x_3 \lor \neg x_3)
\]

graf ten będzie mieć 6 wierzchołków: \(x_1, \neg x_1, x_2, \neg x_2, x_3, \neg x_3\).

Rozpiszmy sobie teraz tę formułę z przykładu korzystając z naszej wcześniejszej obserwacji:

\[
    (x_1 \lor \neg x_2) \land (x_3 \lor \neg x_3) \equiv (\neg x_1 \implies \neg x_2) \land (x_2 \implies x_1) \land (\neg x_3 \implies \neg x_3) \land (x_3 \implies x_3).
\]

,,Strzałki implikacji'' zostają krawędziami w naszym grafie. Tak jest -- konstruujemy graf implikacji literałów. 

\begin{center}

\begin {tikzpicture}[-latex ,auto ,node distance =4 cm and 5cm ,on grid , semithick ,
literal/.style ={ circle, top color=white, draw, text=blue , minimum width =1 cm, bottom color=pink}]
\node[literal] (X1) {\(x_1\)};
\node[literal] (X2) [right=of X1] {\(x_2\)};
\node[literal] (X3) [right=of X2] {\(x_3\)};
\node[literal] (Y1) [below=of X1] {\(\neg x_1\)};
\node[literal] (Y2) [below=of X2] {\(\neg x_2\)};
\node[literal] (Y3) [below=of X3] {\(\neg x_3\)};

\path (Y1) edge [bend left=25] (Y2);
\path (X2) edge [bend left=25] (X1);
\path (Y3) edge [loop left] (Y3);
\path (X3) edge [loop left] (X3);


\end{tikzpicture}
\end{center}

W tym grafie wyznaczamy silnie spójne składowe (jest to graf skierowany, rzecz jasna) i zastanówmy się jaką semantykę ma silnie spójna składowa w takim grafie. 

Otóż -- z definicji -- w silnie spójnej składowej dla każdej pary wierzchołków do niej należących istnieje ścieżka. W naszym przypadku, oznacza to istnienie ciągu implikacji pomiędzy dowolną parą literałów (w jedną i drugą stronę). A to oznacza, że wszystkie literały w obrębie jednej silnie spójnej składowej są sobie \textbf{równoważne}. Innymi słowy -- w obrębie danej spójnej składowej wszystkie literały muszą być zwartościowane tak samo jeśli formuła ma być prawdziwa. 

Nazwijmy graf implikacji \(G\), a formułę którą dostaliśmy na wejściu (i z której on powstał) \(\varphi\). Wykonujemy fajną obserwację:

\begin{lemma}
\(\varphi\) jest spełnialna wtedy i tylko wtedy, gdy w żadnej silnie spójnej składowej \(G\) nie występuje zmienna wraz ze swoim zaprzeczeniem. 
\end{lemma}
\begin{proof}
Jeśli w obrębie danej silnie spójnej składowej występuje zmienna wraz ze swoim zaprzeczeniem, to w oczywisty sposób \(\varphi\) spełnialna nie będzie. 

Zastanówmy się co się dzieje w sytuacji, gdy nie ma takiej silnie spójnej składowej w której występuje zmienna wraz ze swoim zaprzeczeniem. Otóż wtedy jest fajnie.

Wykonujemy standardowy trik i kompresujemy każdą silnie spójną składową do jednego wierzchołka; tak otrzymany graf silnie spójnych składowych jest DAGiem\footnote{Oczywiste.}. Skoro jest to DAG, możemy wykonać sortowanie topologiczne. Idziemy po tym grafie w odwrotnym porządku topologicznym i wartościujemy wierzchołki tego grafu (co ma taką interpretację, że wszystkie literały w obrębie danej silnie spójnej składowej są tak wartościowane). Jeśli dany wierzchołek nie ma przypisanego wartościowania to wartościujemy go na prawdę, a wierzchołek zawierający zanegowane zmienne od nas wartościujemy od razu na fałsz. 

Trzeba niestety (siłą rzeczy) udowodnić, że ta procedura zawsze da nam prawidłowe wartościowanie.

Pierwsza obserwacja: dla dowolnej silnej spójnej składowej \(S\), jeżeli ,,dostaje'' ona wartościowane na prawdę, to istnieje \textbf{dokładnie} jedna silnie spójna składowa która zostanie zwartościowana na fałsz. Załóżmy nie wprost, że tak nie jest:

\begin{enumerate}
    \item Jeśli zwartościowaliśmy na fałsz więcej niż jedną spójną, to oznaczałoby że mamy literały \(a, b\) takie, że \(a \iff b\), ale \(\neg a \notiff \neg b\), co jest sprzeczne z tym jak konstruowaliśmy ten graf. 
    \item Analogiczny argument stosujemy by wykazać, że ta spójna którą obecnie chcemy zwartościować na fałsz nie była już tak wartościowana wcześniej podczas działania algorytmu. 
\end{enumerate}

Z tego mamy bijekcję między silnie spójnymi składowymi (na spójne które zawierają jakieś literały i spójne które zawierają dokładnie negacje tych literałów). 

Teraz pokażemy, że algorytm przypisywania wartości działa poprawnie, to znaczy nie będzie takiej sytuacji gdzie po przypisaniu wartości ,,prawda'' do jakiegoś wierzchołka będzie szła z niego krawędź do wierzchołka który już został zwartościowany na ,,fałsz''. 

Załóżmy nie wprost, że doszło do takiej sytuacji: mamy zatem literały \(x\) i \(y\) takie, że \( x \implies y\), ale wartość \(y\) została ustawiona przez algorytm na fałsz, a \(x\) na prawdę. Zauważamy, że (ze wcześniejszej obserwacji) \(\neg y\) zostało na pewno wcześniej zwartościowane na ,,prawdę'', oraz \( \neg y \implies \neg x\) z definicji naszego grafu. Ale przecież skoro teraz wartościujemy spójną składową z \(x\), to jest ona później w porządku topologicznym niż spójna z \( \neg y\) i wcześniej niż spójna z \( \neg x\), a więc ścieżka z \( \neg y\) do \( \neg x\) istnieć nie może. Otrzymana sprzeczność pokazuje, że do takiej sytuacji dojść nie może. \end{proof}

Innymi słowy, by stwierdzić czy \textsc{2-SAT} jest spełnialny wystarczy sprawdzić czy istnieje jakaś ,,wewnętrznie sprzeczna'' silnie spójna składowa w \(G\). Aby explicite otrzymać wartościowanie, możemy użyć podanego algorytmu. Oczywiście, taki check (jak i ten algorytm) są wielomianowe, więc wszystko jest super. 

\end{proof}