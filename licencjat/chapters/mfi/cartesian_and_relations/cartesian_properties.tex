Iloczyn kartezjański ma parę przyjemnych własności przy działaniu na zbiorach w~konkretnej postaci.
\begin{itemize}
    \item \(x \times \emptyset = \emptyset\)
        \begin{proof}
            Inkluzja \(x \times \emptyset \supseteq \emptyset\) jest trywialna. Wystarczy pokazać inkluzję w~drugą stronę, tzn. \(z \in x \times \emptyset \implies z \in \emptyset\). 
            
            Jeśli założymy, że \(z \in x \times \emptyset\), to z~definicji \(\exists_a\exists_b z = \pars{a, b}\) oraz \(a \in x \land b \in \emptyset\). Możemy zatem spokojnie postawić sobie implikację
            \begin{equation*}
                a \in x \land b \in \emptyset \implies z \in \emptyset
            \end{equation*}
            ponieważ jej poprzednik jest fałszywy, a~do fałszywego poprzednika możemy dołożyć dowolny następnik, jaki nam się podoba.
        \end{proof}
    \item \(x \times \pars{y \cup z} = \pars{x \times y} \cup \pars{x \times z}\)
        \begin{proof}
            Przejdziemy tutaj raz przez prawdziwe królestwo kwantyfikatorów, a~w~kolejnych podpunktach będziemy już dowodzić bardziej intuicyjnie.
            \begin{equation*}
                \begin{split}
                    \xi \in x \times \pars{y \cup z}
                        \iff& \exists_{a \in x}\exists_{b \in y \cup z} \xi = \pars{a, b}\\
                        \iff& \exists_{a \in x}\exists_b\pars{\pars{b \in y \lor b \in z} \land \xi = \pars{a, b}}\\
                        \iff& \exists_{a \in x}\exists_b\pars{\pars{b \in y \land \xi = \pars{a, b}} \lor \pars{b \in z \land \xi = \pars{a, b}}}\\
                        \iff& \exists_{a \in x}\pars{\exists_b\pars{b \in y \land \xi = \pars{a, b}} \lor \exists_b\pars{b \in z \land \xi = \pars{a, b}}}\\
                        \iff& \pars{\exists_{a \in x}\exists_b\pars{b \in y \land \xi = \pars{a, b}}} \lor \pars{\exists_{a \in x}\exists_b\pars{b \in z \land \xi = \pars{a, b}}}\\
                        \iff& \pars{\exists_{a \in x}\exists_{b \in y}\xi = \pars{a, b}} \lor \pars{\exists_{a \in x}\exists_{b \in z}\xi = \pars{a, b}}\\
                        \iff& \xi \in \pars{x \times y} \lor \xi \in \pars{x \times z}\\
                        \iff& \xi \in \pars{x \times y} \cup \pars{x \times z}
                \end{split}
            \end{equation*}
            Skorzystaliśmy tutaj z~rozdzielności konjunkcji względem alternatywy oraz z~faktu, że
            \begin{equation*}
                \exists_v\pars{\varphi \lor \psi} \iff \pars{\exists_v\varphi} \lor \pars{\exists_v\psi}
            \end{equation*}
        \end{proof}
    \item \(x \times \pars{y \cap z} = \pars{x \times y} \cap \pars{x \times z}\)
        \begin{proof}
            \begin{equation*}
                \begin{split}
                    \pars{a, b} \in x \times \pars{y \cap z}
                        \iff& a \in x \land b \in y \cap z\\
                        \iff& a \in x \land \pars{b \in y \land b \in z}\\
                        \iff& \pars{a \in x \land b \in y} \land \pars{a \in x \land b \in z}\\
                        \iff& \pars{a, b} \in \pars{x \times y} \land \pars{a, b} \in \pars{x \times z}\\
                        \iff& \pars{a, b} \in \pars{x \times y} \cap \pars{x \times z}
                \end{split}
            \end{equation*}
            Skorzystaliśmy tu z~rozdzielności konjunkcji względem niej samej.
        \end{proof}
    \item \(x \times \pars{y \setminus z} = \pars{x \times y} \setminus \pars{x \times z}\)
        \begin{proof}
            \begin{equation*}
                \begin{split}
                    \pars{a, b} \in x \times \pars{y \setminus z}
                        \iff& a \in x \land b \in \pars{y \setminus z}\\
                        \iff& a \in x \land \pars{b \in y \land b \not\in z}\\
                        \iff& \pars{a \in x \land b \in y} \land \pars{a \in x \land b \not \in z}\\
                        \iff& \pars{a, b} \in \pars{x \times y} \land \pars{a, b} \not\in \pars{x \times z}\\
                        \iff& \pars{a, b} \in \pars{x \times y} \setminus \pars{x \times z}
                \end{split}
            \end{equation*}
            Ponownie skorzystaliśmy tu z~rozdzielności konjunkcji względem niej samej.
        \end{proof}
\end{itemize}
Ponadto, iloczyn kartezjański jest też \emph{monotoniczny} ze względu na każdą współrzędną. Zgadza się to z~intuicją --- jeśli na jednej ze współrzędnych mamy ,,więcej'' możliwości wyboru elementu, to również zbiór możliwych par będzie ,,większy''.
\begin{itemize}
    \item \(x \subseteq y \implies \pars{x \times z} \subseteq \pars{y \times z}\)
        \begin{proof}
            Załóżmy, że \(\pars{a, b} \in \pars{x \times z}\). Oznacza to, że \(a \in x\) i~\(b \in z\). Skoro \(a \in x\) oraz \(x \subseteq y\), to \(a \in y\). Skoro zatem \(a \in y\) i~\(b \in z\), to \(\pars{a, b} \in \pars{y \times z}\). To dowodzi postulowanej inkluzji.
        \end{proof}
    \item \(x \subseteq y \implies \pars{z \times x} \subseteq \pars{z \times y}\)
        \begin{proof}
            Analogiczny jak dla pierwszej współrzędnej.
        \end{proof}
\end{itemize}