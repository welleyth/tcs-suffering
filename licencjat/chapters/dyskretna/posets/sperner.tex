    \begin{theorem}[Twierdzenie Spernera]
       Najdłuższy antyłańcuch \(F\) w kracie zbiorów $B_n$ ma moc $\binom{n}{\lfloor\frac{n}{2}\rfloor} = \binom{n}{\lceil\frac{n}{2}\rceil}$
    \end{theorem}

    \begin{proof}
        Jesteśmy w stanie wskazać antyłańcuch takiej długości; jest to po prostu antyłańcuch w ,,warstwie'' $\lfloor n/2 \rfloor$ kraty $B_n$, tzn. wszystkie podzbiory $[n]$ mocy $\lfloor n/2 \rfloor$. Pokazujemy teraz, że nie istnieje dłuższy antyłańcuch, korzystając z faktu, że:
       \begin{equation*}
            \binom{n}{k} \leq \binom{n}{\lfloor\frac{n}{2}\rfloor} = \binom{n}{\lceil\frac{n}{2}\rceil}
       \end{equation*}
       a więc z nierówności LYM mamy, że:
       \begin{equation*}
           \sum_{k=0}^{n}\frac{f_k}{\binom{n}{\lfloor n/2 \rfloor}} \leq \sum_{k=0}^{n}\frac{f_k}{\binom{n}{k}} \leq 1
       \end{equation*}
       co dowodzi tezę bo w takim razie:
       \begin{equation*}
           \sum_{k=0}^{n}\frac{f_k}{\binom{n}{\lfloor n/2 \rfloor}} = 
           |F| \cdot \frac{1}{\binom{n}{\lfloor n/2 \rfloor}} \leq 1
        \end{equation*}
        więc 
        \begin{equation*}
           |F| \leq \binom{n}{\lfloor n/2 \rfloor} =
           \binom{n}{\lfloor n/2 \rfloor}
       \end{equation*}
       \end{proof}