    \begin{theorem}[Oczywiste twierdzenie o przekrojach]
        Dla dowolnego przepływu w sieci przepływowej $f$ i dla dowolnego jej przekroju $(S,T)$ jest tak, że przepływ przez przekrój jest słabo mniejszy niż jego przepustowość (tzn. $f(S,T) \leq c(S,T)$).
    \end{theorem}

    \begin{proof}
        To widać. W sensie serio, gdyby suma po przepływach krawędzi wychodzących z $S$ była większa niż suma po ich przepustowościach, to znaczyłoby że coś gdzieś poszło bardzo mocno nie tak.
    \end{proof}

    \begin{theorem}[Mniej oczywiste twierdzenie o przekrojach]
        Przepływ przez każdy przekrój sieci jest taki sam i wynosi $val(f)$.
    \end{theorem}

    \begin{proof}
        Indukcja po liczbie wierzchołków w części $S$ przekroju (tej do której należy źródło). Przypadek bazowy gdy $|S| = 1$ trywialny. W przypadku ogólnym mamy sobie jakiś przekrój $(S,T)$. Weźmy teraz jakiś $x \in S$ (różny od źródła, na pewno taki jest bo dla przypadku gdzie jest tylko jeden wierzchołek w $S$ mamy to już udowodnione) i wrzućmy go do $T$, otrzymując alternatywny przekrój $(S', T')$. Teraz jest śmiesznie, bo wiemy z założenia indukcyjnego że przepływ przez $(S', T') = val(f)$, bo $S'$ ma mniejszą moc od $S$. Rozpiszmy sobie teraz $f(S', T')$ oraz $f(S,T)$ (przez $v_{s}$ etc. oznaczam wierzchołek należący do $S$ lub innych zbiorów):
        \begin{equation*}
            f(S,T) = \sum_{(v_s, v_t) \in E, v_s \not = x} f(v_s, v_t) + \sum_{(x, v_t) \in E} f(x, v_t) - \sum_{(v_t,v_s) \in E, v_s \not = x} f(v_t,v_s) - \sum_{(v_t,x) \in E} f(v_t,x)
        \end{equation*}
        Jako że jeśli punkt $v \in S$ oraz $v \not = x$, to $v \in S'$, możemy to uprościć i zapisać jako:
        \begin{equation*}
            f(S,T) = \sum_{(v_{s'}, v_t) \in E} f(v_{s'}, v_t) + \sum_{(x, v_t) \in E} f(x, v_t) - \sum_{(v_t,v_{s'}) \in E} f(v_t,v_{s'}) - \sum_{(v_t,x) \in E} f(v_t,x)
        \end{equation*}
        \begin{equation*}
            f(S',T') = \sum_{(v_{s'}, v_{t'}) \in E, v_{t'} \not = x} f(v_{s'}, v_{t'}) + \sum_{(v_{s'}, x) \in E} f(v_{s'}, x) - \sum_{(v_{t'},v_{s'}) \in E, v_{t'} \not = x} f(v_{t'},v_{s'}) - \sum_{(x,v_{s'}) \in E} f(x,v_{s'})
        \end{equation*}
        Jako, że jeśli punkt $v \in T'$ oraz $v \not = x$, to wiemy że $v \in T$: 
        \begin{equation*}
            f(S',T') = \sum_{(v_{s'}, v_{t}) \in E} f(v_{s'}, v_{t}) + \sum_{(v_{s'}, x) \in E} f(v_{s'}, x) - \sum_{(v_{t},v_{s'}) \in E} f(v_{t},v_{s'}) - \sum_{(x,v_{s'}) \in E} f(x,v_{s'})
        \end{equation*}
        To wygląda przerażająco, ale w sumie wynika wprost z definicji przepływu, po prostu dodajemy te wszystkie krawędzie do siebie. Zasadniczo nic ciekawego. Co teraz jest fajne to to, że możemy policzyć $f(S,T) = f(S',T')$ (zwróćmy uwagę, że pierwszy i trzeci składnik się uproszczą (co w sumie ma sens, bo jedyne co zmienialiśmy to pozycja $x$, więc to co przepływa w reszcie grafu się nie zmieniło -- ależ plot twist) , zostawiając tylko składniki z $x$).

        \begin{equation*}
            f(S,T) - f(S',T') = \sum_{(x, v_t) \in E} f(x, v_t) - \sum_{(v_{s'}, x) \in E} f(v_{s'}, x) - \sum_{(v_t,x) \in E} f(v_t,x) + \sum_{(x,v_{s'}) \in E} f(x,v_{s'})
        \end{equation*}
        Jako, że do $x$ wpływa tyle samo co wypływa:
        \begin{equation*}
            \begin{split}
                f(S,T) - f(S',T')
                    &= \sum_{(x, v_t) \in E} f(x, v_t) - \sum_{(v_t,x) \in E} f(v_t,x) + \sum_{(x,v_{s'}) \in E} f(x,v_{s'}) - \sum_{(v_{s'}, x) \in E} f(v_{s'}, x)\\
                    &= 0 + 0 = 0
            \end{split}
        \end{equation*}

        Skąd mamy szokujące odkrycie, że $f(S,T) = f(S',T')$. Ale przecież $f(S',T')$ było równe $val(f)$, czyli wszystko się zgadza.
    \end{proof}
