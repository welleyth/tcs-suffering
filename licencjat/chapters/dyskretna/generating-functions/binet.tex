 \begin{theorem}[Wzór Bineta]
        \begin{equation}
            f_n = \frac{1}{\sqrt{5}} \cdot \left( \left(\frac{1 + \sqrt{5}}{2}\right)^{n} - \left(\frac{1 - \sqrt{5}}{2}\right)^{n} \right)
        \end{equation}
    \end{theorem}

    \begin{proof}
        Rozpisujemy sobie funkcję tworzącą ciągu $f_n$:

        \begin{equation*}
            F(x) = f_0 + f_1 \cdot x + f_2 \cdot x^2 + f_3 \cdot x^3 \dots = 
        \end{equation*}
        \begin{equation*}
           = f_0 + f_1 \cdot x + (f_0 + f_1) \cdot x^2 + (f_1 + f_2) \cdot x^3 + \dots =
        \end{equation*}
        \begin{equation*}
           = f_0 + f_1 \cdot x + f_0 \cdot x^2 + f_1 \cdot x^2 + f_1 \cdot x^3 + f_2 \cdot x^3 + \dots =
        \end{equation*}
          \begin{equation*}
           = f_0 + f_1 \cdot x + f_0 \cdot x^2 + f_1 \cdot x^3 + \dots + f_1 \cdot x^2 +  f_2 \cdot x^3 + \dots =
        \end{equation*}
         \begin{equation*}
           = f_0 + f_1 \cdot x + x^2 \cdot (f_0 + f_1 \cdot x + \dots) + x \cdot (f_1 \cdot x +  f_2 \cdot x^2 + \dots) =
        \end{equation*}
        \begin{equation*}
           = f_0 + f_1 \cdot x + x^2 \cdot F(x) + x \cdot (F(x) - f_0) =
        \end{equation*}
        \begin{equation*}
           = 0 + 1 \cdot x + x^2 \cdot F(x) + x \cdot (F(x) - 0) =
        \end{equation*}
         \begin{equation*}
           = x + x^2 \cdot F(x) + x \cdot F(x)
        \end{equation*}

        W takim razie mamy, że:
        \begin{equation*}
            F(x) = x + x^2 \cdot F(x) + x \cdot F(x)
        \end{equation*}
        \begin{equation*}
            F(x) -  x^2 \cdot F(x) - x \cdot F(x)  = x
        \end{equation*}
         \begin{equation*}
            F(x) \cdot (1 - x^2 - x) = x
        \end{equation*}
           \begin{equation*}
            F(x) = \frac{x}{-x^2 -x + 1}
        \end{equation*}

        Mianownik możemy rozbić (za pomocą liczenia jakichś delt czy coś): 
        \begin{equation*}
            F(x) = \frac{x}{(-1) \cdot \left(x - \left(- \frac{1 + \sqrt{5}}{2}\right)\right) \cdot \left(x - \left(- \frac{1 - \sqrt{5}}{2}\right)\right)}
        \end{equation*}

        Nie no, serio, jeśli ktoś myśli że będę TeXować te przekształcenia to się myli. Powinno wyjść po przekształceniach że:
        \begin{equation*}
            F(x) = \frac{x}{(1-ax) \cdot (1-bx)}
        \end{equation*}
        gdzie $a = \frac{1 + \sqrt{5}}{2}, b=\frac{1 - \sqrt{5}}{2}$

        Dalej rozbijamy na ułamki proste:
        \begin{equation*}
            F(x) = \frac{A}{1-ax} + \frac{B}{1-bx}
        \end{equation*}
        $A$ powinno wyjść $\frac{1}{\sqrt{5}}$, $B$ powinno wyjść $- \frac{1}{\sqrt{5}}$.

        Odwijamy każdą z tych funkcji tworzących z osobna, korzystając ze wzoru podanego we wcześniejszym rozdziale i otrzymujemy wzór. 
    \end{proof}