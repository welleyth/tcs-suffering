\textbf{Algebra relacyjna} to zbiór relacji wraz z określonymi na nich operatorami.
Każdy z operatorów działa na jednej lub wielu relacjach i zwraca nową relację.

Podstawowe operatory (wprowadzone oryginalnie przez Codda w 1970) to:
\begin{itemize}
    \item \textbf{Selekcja} (\( \sigma \)) -- wybiera wiersze, które spełniają warunek.
    \item \textbf{Projekcja} (\( \pi \))-- wybiera kolumny, które są wymagane.
    \item \textbf{Suma} (\( \cup \)) -- łączy dwie relacje.
    \item \textbf{Przecięcie} (\( \cap \)) -- zwraca wspólne wiersze dwóch relacji.
    \item \textbf{Różnica} (\( - \)) -- zwraca różnicę dwóch relacji.
    \item \textbf{Iloczyn kartezjański} (\( \times \)) -- zwraca iloczyn kartezjański dwóch relacji.
    \item \textbf{Złączenie} (\( \bowtie \)) -- zwraca iloczyn kartezjański dwóch relacji, a następnie usuwa duplikaty.
\end{itemize}

Operatory można składać ze sobą, tworząc w ten sposób bardziej skomplikowane działania.

\subsection*{Operatory}

\subsubsection*{Selekcja}

Operator \textbf{selekcji} (\( \sigma \)) wybiera wiersze, które spełniają zadany predykat.

\[
    \sigma_{\text{predykat}}(R)
\]

Jako predykat dopuszczalne są operatory porównania (np. \( =, \neq, <, >, \leq, \geq \)), operatory logiczne (np. \( \land, \lor, \lnot \)) oraz operatory zbiórów (np. \( \in, \notin \)) ze stałymi lub z wartoścami innych atrybutow.

Predykaty mogą być złożone, tj. być koniunkcjami, alternatywami lub zaprzeczeniami innych predykatów.

\subsubsection*{Projekcja}

Operator \textbf{projekcji} (\( \pi \)) wybiera z danej relacji określony zbior atrybutow.

\[
    \pi_{\text{atrybuty}}(R)
\]

Kolejność atrybutów może być dowolna.

Wybierane atrybuty można obkładać dodatkowymi funkcjami.

\subsubsection*{Operatory teoriomnogościowe}

\subsubsection*{Suma}

Operator \textbf{sumy} (\( \cup \)) łączy dwie relacje, zwracając relację, która zawiera wszystkie wiersze z obu relacji.
