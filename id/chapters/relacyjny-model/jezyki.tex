Języki manipulacji danymi w systemach baz danych można podzielić na dwa rodzaje: \textbf{proceduralne} i \textbf{nieproceduralne}.
W przypadku języków proceduralnych użytkownik opisuje nie tylko to, jakie dane chciałby otrzymać, ale dodatkowo podaje określoną strategię wyszukiwania
i uzyskiwania danych na pewnym poziomie szczegółowości.

Przykładem języka proceduralnego jest \textbf{algebra relacyjna}.

W przypadku języków nieproceduralnych użytkownik podaje jedynie warunki, jakie powinien spełniać żądany przez niego wynikowy zbiór danych, nie podaje natomiast metody ich uzyskania.

Przykładem języka nieproceduralnego jest \textbf{rachunek relacyjny}.